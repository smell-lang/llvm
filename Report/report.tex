\documentclass{book}
\usepackage{graphicx}
\usepackage[executivepaper,margin=1in]{geometry}
\usepackage[charter]{mathdesign}
\usepackage{ragged2e}
%-----------------------------------------------------------------
\begin{document}
% ----------------------------------------------------------------
\begin{titlepage}
\begin{center}
 {\huge\bfseries A LLVM Back-end for MLton \\}
 % ----------------------------------------------------------------
 
 \vspace{1.5cm}
 by
 \vspace{0.02cm}
 
 \vspace{1.0cm}
 {\Large\bfseries Rahul Dhawan}\\[5pt]
 rdhawan201455@gmail.com\\[14pt]
  % ----------------------------------------------------------------
 \vspace{2cm}
{Report  submitted to} \\[5pt]
\emph{{IIT Palakkad}}\\[2cm]
{in partial fulfilment for the award of the degree
 of} \\[2cm]
\textsc{\Large{{Bachelor of Technology }}} \\[5pt]
{in Computer Science} \vspace{0.4cm} \\[1cm]

\textsc{\Large{{Supervised by}}}   \\[5pt]
{Dr. Piyush P. Kurur} \vspace{0.4cm} \\[2cm]

% {By}\\[5pt] {\Large \sc {Me}}
 \vfill
 % ----------------------------------------------------------------
{Department of CSE}\\[5pt]
{Ahalia Integrated Campus,}\\[5pt]
{ Palakkad Dist., Kozhippara, Kerala 678557,
 INDIA}\\
 \vfill
{April 2019}
\end{center}
\end{titlepage}
% ----------------------------------------------------------------

\thispagestyle{plain}
\begin{center}
    
    \huge{\textbf{Abstract}}
 
    \vspace{0.4cm}
    \large
    \textbf{An LLVM Back-end for MLton}
    
    \vspace{0.4cm}
    \textbf{Rahul Dhawan}
 
    \vspace{0.4cm}
    \textbf{Supervising Professor: Dr. Piyush P. Kurur }
    
    \vspace{1.0cm}
    \justify
    This report presents the design and implementation of a new LLVM back-end for the ML-ton Standard ML compiler. The motivation of this project is to utilize the features that an LLVM back-end can provide to a compiler, and compare its implementation to the existing back-ends that ML-ton has for C and native assembly (x86 and amd64).  The LLVM back-end was found to offer a greatly simpler implementation compared to the existing back-ends, along with comparable compile times and performance of generated executable, with the LLVM-compiled version performing the best on many of the benchmarks. The LLVM IR can used for optimization of source code and JIT(Just In Time).
\end{center}

\tableofcontents

\chapter{Introduction}
   \Large
   Today's world of computing is undergoing a constant, yet silent revolution. Both
   software and hardware systems - from embedded micro-controllers through to massively
   parallel high-performance computers - are experiencing an ever-increasing degree of
   sophistication and complexity. Although it never actually comes to the fore, compiler
   technology plays a central role in this revolution as the junction between software and
   hardware. Traditionally, a compiler has to fulfill three main requirements:
   \begin{itemize}
   	\item It has to generate efficient code for the target platform.
   	\item It should consume only a reasonable amount of time and memory.
   	\item It must be reliable.
   \end{itemize}

Compilers are generally architected in a three-phase design: The front-end, optimizer
and back-end, shown in figure 1.1. The front-end is responsible for the lexing, parsing, and
type checking of the source code, transforming it into an abstract syntax tree (AST), which
acts as an intermediate representation (IR) in the compiler. The optimizer improves the
efficiency and performance of this intermediate representation by transforming the code to
simpler yet semantically equivalent versions, possibly using different representations if it
is useful to do so. In the back-end, the code is emitted to an executable form, usually as
either machine code that can be directly run on hardware, or byte code that can be run on a
virtual machine.\linebreak \linebreak 
\begin{figure}
	\includegraphics[width=\linewidth]{/home/breezy/Downloads/Drive.png}
	\caption{General compiler pipeline}
	\label{fig:1.1}
\end{figure} Because of the split between the different components of a compiler, it is possible for
multiple compilers for different languages to share the same back-end system. One of the
long-standing issues in compiler design is how to best solve the challenges in utilizing a
common back-end technology. Compiler developers want to ensure the compiled programs
perform the best that they can, but they also want to leverage language-agnostic tools and
libraries that free them from being concerned with the low level details required to make the compiler generate high-performance executables.  One such project that solves this issue
effectively is LLVM, which defines a high-level, target independent assembly language
that can be aggressively optimized and compiled to several different architectures 

 This  report  examines  an  LLVM  back-end  for  MLton and conversion of LLVM AST into the LLVM IR source code. MLton5
is written primarily in SML with a runtime system written in C, and is able to self-host.
MLton's  features  include  support  for  a  large  variety  of  platforms  and  architectures,  the
ability to handle large, resource intensive programs, and aggressive compiler optimizations
that lead to efficient programs with fast running times. The MLton compiler currently has
three back-ends:  C, x86, and amd64.  The C back-end emits the compiled program as C
code, and uses an external C compiler to compile to native code. The x86 and amd64 back-
ends, known together as the native back-ends, emit assembly language directly which is
then assembled by the system assembler into native code. The native back-ends offer better
performance and compile times, but have a limited set of supported platforms.

In this report, we will be looking at the design and implementation of the new back-
end for the MLton Standard ML compiler, using the LLVM IR as a target and Design and implementation of an algorithm to convert the LLVM ast into LLVM IR.

The rest of the report is organized as follows. Chapter 2 goes over the design of MLton
and the techniques it uses to compile Standard ML programs.  It also goes over the design
of LLVM, and how it can be used as a compiler back-end.  Chapter 3 describes the design
and implementation of the LLVM back-end, going over the design choices made and the
strategy used for translating to LLVM. Chapter 4 goes over the design and implementation of the algorithm to convert the LLVM AST into LLVM IR. Chapter 5 gives concluding remarks for this project and ideas for further work

\chapter{Background}
\Large
	\section{Back-end Targets}
	A  big  challenge  in  the  design  of  compilers  for  high-level  languages  is  finding  the  best
	way to implement the compiler's back-end.  Compiler developers want to use a technique
	that allows the generated executable to run as efficiently as possible, as that is an important
	factor in judging the quality of the compiler. However, they also want to minimize the effort
	in implementing the back-end, ideally by sharing infrastructure with other compilers. Due
	to many compiler writers prioritizing the former, this challenge has led to a situation where
	the popular implementations of languages like C, Java, Python, and Haskell share little or
	nothing in common.
	
	A  big  challenge  in  the  design  of  compilers  for  high-level  languages  is  finding  the  best
	way to implement the compiler’s back-end.  Compiler developers want to use a technique
	that allows the generated executables to run as efficiently as possible, as that is an important
	factor in judging the quality of the compiler. However, they also want to minimize the effort
	in implementing the back-end, ideally by sharing infrastructure with other compilers. Due
	to many compiler writers prioritizing the former, this challenge has led to a situation where
	the popular implementations of languages like C, Java, Python, and Haskell share little or
	nothing in common.
	
	Major factor in the design of back-end of any Language is the kind of language the back-end will be going to target.
	achievable targets can be categorized into four categories:
	
	\begin{itemize}
		\item \textbf{Naive Assembly} :  This is one of the genuine preferred choice, as it offers the compiler
		writer the most control over how the compiled executable is written, and it control
		the dependence on external tools as all you need is the system assembler. However, this
		way takes the most amount of effort by the compiler writers to implement and
		maintain.  Assembly languages are complex and disclose many low level details, so
		it takes a considerable amount of effort to write an effective implementation.  Also,
		an assembly back-end is target specific, so adding support for a new architecture in a
		compiler requires a lot of effort.
		.
		\item \textbf{High Level Language} : Compilers can produce to a different high-level language, using
		compilers for that language as foreign tools to complete the compilation process.
		Most often the language is C ,  because it is low level enough to not interfere
		too much with the semantics of the source language or final IR of the compiler, and
		because the language has attractive performance and limited runtime overhead.  Also,
		this way allocate flexibility for free due to the presence of C compilers across most
		computing platforms. This approach still has its bugs, due to lack of fine control of
		code generation details such as tail calls, and longer compilation times due to having
		to parse and compile source again as part of the back-end stage. 
		\item \textbf{Managed virtual environments} : A popular choice for programming languages in the
		past couple of decades is to compile to a high-level and portable
		bytecode format, and at run-time have it execute on a virtual machine. This has been
		the choice of execution model for modern compiled languages like Java and C sharp, and
		for  scripting  languages  like  Python  and  Lua.   To  help  overcome  the  performance
		penalty of executing on an interpreter,  the virtual machines often use just-in-time
		(JIT) compilation which compiles the executing code to native machine code as it
		gets executed. The benefits of this technique include portability on all platforms the
		virtual machine runs on, and allowing code to take advantage of existing libraries and
		rich runtime features provided by the platform such as garbage collection and excep-
		tion handling.   However,  this approach generally suffers from worse performance
		compared to compiling to native assembly, and can raise issues when the runtime
		features of the platform do not match up nicely to the runtime features needed by the
		language.  For example, the garbage collection techniques may not work well based
		on the style and frequency in which the language allocates objects, or the exception
		handling system on the virtual machine may not match the semantics of exceptions
		in the language.
		\item \textbf{High Level Assembly} : The final alternative is High level assembly.  High level assembly languages are
		low-level enough to not interfere with the abstractions in high level programming
		languages,  but  also  high-level  enough  to  abstract  away  the  very  low  level  details
		of assembly language such as register allocation and instruction scheduling.  They
		also have the ability to be optimized in both target-independent and target dependent
		ways, producing high-performance executables. One such language that implements
		all of these features is the LLVM IR.
	\end{itemize}
	


\end{document}